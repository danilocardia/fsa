% ------------------------------------------------------------------------
% ------------------------------------------------------------------------
% abnTeX2: Modelo de Trabalho Academico (tese de doutorado, dissertacao de
% mestrado e trabalhos monograficos em geral) em conformidade com 
% ABNT NBR 14724:2011: Informacao e documentacao - Trabalhos academicos -
% Apresentacao
% ------------------------------------------------------------------------
% ------------------------------------------------------------------------

\documentclass[
	% -- opções da classe memoir --
	12pt,					% tamanho da fonte
	openright,		% capítulos começam em pág ímpar (insere página vazia caso preciso)
	%twoside,			% para impressão em verso e anverso. Oposto a oneside
	oneside,			% para impressão em verso e anverso. Oposto a oneside
	a4paper,			% tamanho do papel. 
	% -- opções da classe abntex2 --
	%chapter=TITLE,		% títulos de capítulos convertidos em letras maiúsculas
	%section=TITLE,		% títulos de seções convertidos em letras maiúsculas
	%subsection=TITLE,	% títulos de subseções convertidos em letras maiúsculas
	%subsubsection=TITLE,% títulos de subsubseções convertidos em letras maiúsculas
	% -- opções do pacote babel --
	english,			% idioma adicional para hifenização
	french,				% idioma adicional para hifenização
	spanish,			% idioma adicional para hifenização
	brazil,				% o último idioma é o principal do documento
	]{abntex2}

% ---
% Pacotes fundamentais 
% ---
\usepackage{cmap}				% Mapear caracteres especiais no PDF
\usepackage{lmodern}			% Usa a fonte Latin Modern		

\usepackage[T1]{fontenc}
\usepackage[utf8]{inputenc}	
%\usepackage[T1]{fontenc}		% Selecao de codigos de fonte.
%\usepackage[latin1]{inputenc}		% Codificacao do documento (conversão automática dos acentos)
\usepackage{lastpage}			% Usado pela Ficha catalográfica
\usepackage{indentfirst}		% Indenta o primeiro parágrafo de cada seção.
\usepackage{color}				% Controle das cores
\usepackage{graphicx}			% Inclusão de gráficos
% ---
		
% ---
% Pacotes adicionais, usados apenas no âmbito do Modelo Canônico do abnteX2
% ---
\usepackage{lipsum}				% para geração de dummy text
% ---

% ---
% Pacotes de citações
% ---
\usepackage[brazilian,hyperpageref]{backref}	 % Paginas com as citações na bibl
\usepackage[alf]{abntex2cite}	% Citações padrão ABNT

% --- 
% CONFIGURAÇÕES DE PACOTES
% --- 

% ---
% Configurações do pacote backref
% Usado sem a opção hyperpageref de backref
\renewcommand{\backrefpagesname}{Citado na(s) página(s):~}
% Texto padrão antes do número das páginas
\renewcommand{\backref}{}
% Define os textos da citação
\renewcommand*{\backrefalt}[4]{
	\ifcase #1 %
		Nenhuma citação no texto.%
	\or
		Citado na página #2.%
	\else
		Citado #1 vezes nas páginas #2.%
	\fi}%
% ---

% ---
% Informações de dados para CAPA e FOLHA DE ROSTO
% ---
\titulo{Otimizador de tráfego\\por malha de semáforos adaptáveis}
\autor{Danilo Cardia de Oliveira\\João Henrique Silva Suniga\\Murilo Guilherme Lopes}
\local{Brasil}
\data{2014, v-1.0}
\orientador{Jacinto Cansado}
\coorientador{}
\instituicao{%
  Centro Universitário Fundação Santo André - FSA
  \par
  Faculdade de Engenharia
  "Engenheiro Celso Daniel"}
\tipotrabalho{TCC}
% O preambulo deve conter o tipo do trabalho, o objetivo, 
% o nome da instituição e a área de concentração 
\preambulo{Monografia apresentada ao Programa de Graduação em Engenharia da Computação do
					Centro Universitário Fundação Santo André, como requisito
					parcial para obtenção do título de Engenheiro da Computação
					}
% ---


% ---
% Configurações de aparência do PDF final

% alterando o aspecto da cor azul
\definecolor{blue}{RGB}{41,5,195}

% informações do PDF
\makeatletter
\hypersetup{
     	%pagebackref=true,
		pdftitle={\@title}, 
		pdfauthor={\@author},
    	pdfsubject={\imprimirpreambulo},
	    pdfcreator={LaTeX with abnTeX2},
		pdfkeywords={abnt}{latex}{abntex}{abntex2}{trabalho acadêmico}, 
		colorlinks=true,       		% false: boxed links; true: colored links
    	linkcolor=black,          	% color of internal links
    	citecolor=black,        		% color of links to bibliography
    	filecolor=black,      		% color of file links
		urlcolor=black,
		bookmarksdepth=4
}
\makeatother
% --- 

% --- 
% Espaçamentos entre linhas e parágrafos 
% --- 

% O tamanho do parágrafo é dado por:
\setlength{\parindent}{1.3cm}

% Controle do espaçamento entre um parágrafo e outro:
\setlength{\parskip}{0.2cm}  % tente também \onelineskip

% ---
% compila o indice
% ---
\makeindex
% ---

% ----
% Início do documento
% ----
\begin{document}

% Retira espaço extra obsoleto entre as frases.
\frenchspacing 

% ----------------------------------------------------------
% ELEMENTOS PRÉ-TEXTUAIS
% ----------------------------------------------------------
% \pretextual

% ---
% Capa
% ---
\imprimircapa
% ---

% ---
% Folha de rosto
% (o * indica que haverá a ficha bibliográfica)
% ---
\imprimirfolhaderosto*
% ---

%\include{fichaCatalografica}

%\include{errata}

% ---
% Inserir folha de aprovação
% ---

% Isto é um exemplo de Folha de aprovação, elemento obrigatório da NBR
% 14724/2011 (seção 4.2.1.3). Você pode utilizar este modelo até a aprovação
% do trabalho. Após isso, substitua todo o conteúdo deste arquivo por uma
% imagem da página assinada pela banca com o comando abaixo:
%
% \includepdf{folhadeaprovacao_final.pdf}
%
\begin{folhadeaprovacao}

  \begin{center}
    {\ABNTEXchapterfont\large\imprimirautor}

    \vspace*{\fill}\vspace*{\fill}
    {\ABNTEXchapterfont\bfseries\Large\imprimirtitulo}
    \vspace*{\fill}
    
    \hspace{.45\textwidth}
    \begin{minipage}{.5\textwidth}
        \imprimirpreambulo
    \end{minipage}%
    \vspace*{\fill}
   \end{center}
    
   Trabalho aprovado. \imprimirlocal, 30 de setembro de 2014:

   \assinatura{\textbf{\imprimirorientador} \\ Orientador} 
   %\assinatura{\textbf{Professor} \\ Convidado 1}
   %\assinatura{\textbf{Professor} \\ Convidado 2}
   %\assinatura{\textbf{Professor} \\ Convidado 3}
   %\assinatura{\textbf{Professor} \\ Convidado 4}
      
   \begin{center}
    \vspace*{0.5cm}
    {\large\imprimirlocal}
    \par
    {\large\imprimirdata}
    \vspace*{1cm}
  \end{center}
  
\end{folhadeaprovacao}
% ---

% ---
% Dedicatória
% ---
\begin{dedicatoria}
   \vspace*{\fill}
   \centering
   \noindent
   \textit{ Este trabalho é dedicado aos nossos familiares que nos 
	apoiaram incondicionalmente. } \vspace*{\fill}
\end{dedicatoria}
% ---

% ---
% Agradecimentos
% ---
\begin{agradecimentos}

Fazer os agradecimentos e créditos. 

\end{agradecimentos}
% ---

% ---
% Epígrafe
% ---
\begin{epigrafe}
    \vspace*{\fill}
	\begin{flushright}
		\textit{A grande tragédia da ciência: \\
		o massacre de uma bela hipótese por parte de um horrível fato.
		\\Thomas Huxley}
	\end{flushright}
\end{epigrafe}
% ---

% Resumo
% resumo em português
\begin{resumo}

	Desenvolver um projeto de uso do processador  Arduino que mostre (sem complexidade) a um usuário (sem o conhecimento em programação) o fundamento do algoritmo de programação. O projeto será desenvolvido tendo a parte teórica, no qual o grupo deverá documentar todo o andamento (cronograma) do projeto, como também a parte prática, no qual o grupo irá primeiramente explorar as funcionalidades da placa Arduino para que sejam implantadas aplicações de controle. A princípio temos como expectativa obter um resultado a partir da sequência de leds acesos e/ou apagados mediante resultado de uma fórmula implementada, e exploração das portas de comunicação como desafio extra para o grupo.

 \vspace{\onelineskip}
    
 \noindent
 \textbf{Palavras-chaves}: Arduino, Algoritmo.
\end{resumo}

% resumo em inglês
\begin{resumo}[Abstract]
 \begin{otherlanguage*}{english}
   
	Develop a project using the Arduino processor that show (no complexity) to a user (without programming skills) the foundation of the programming algorithm. The project will be developed with the theoretical part, in which the whole group will document progress (schedule) of the project, as well as the practical part, in which the group will first explore the features of the Arduino board to control which applications are deployed. At first we expected to get a result from the sequence of lit LEDs and / or cleared by the result  of a formula implemented, and operating communication ports as extra challenge for the group. 

   \vspace{\onelineskip}
 
   \noindent 
   \textbf{Key-words}: Arduino, Algorithm.
 \end{otherlanguage*}
\end{resumo}

% ---
% inserir lista de ilustrações
% ---
\pdfbookmark[0]{\listfigurename}{lof}
\listoffigures*
\cleardoublepage
% ---

% ---
% inserir lista de tabelas
% ---
\pdfbookmark[0]{\listtablename}{lot}
\listoftables*
\cleardoublepage
% ---

% ---
% inserir o sumario
% ---
\pdfbookmark[0]{\contentsname}{toc}
\tableofcontents*
\cleardoublepage
% ---

\textual

\chapter{Introdução}

\section{Motivação}

Os problemas de transporte afetam diretamente a qualidade de vida dos habitantes da cidade de São Paulo, e também, de outras grandes cidades metropolitanas de países em desenvolvimento.
Em horários de pico, a quantidade de veículos nas principais vias da cidade é tão grande que grande parte dos motoristas já se acostumaram a dirigir em engarrafamentos quilométricos, acostumaram-se também, a gastar horas para fazer um trajeto que, em via livre, gastaria-se poucas dezenas de minutos.

Ainda que o tráfego urbano seja o resultado de um complexo conjunto de fatores e que a frota de veículos particulares na cidade de São Paulo tenha aumentado cerca de 1080\% (GEIPOT) nos últimos 30 anos \cite{serra2004aplicaccoes}, não houve uma evolução perceptível no controle de trânsito ou no uso de novas tecnologias pelas entidades de engenharia de tráfego, sendo que, é impressindível o conhecimento sistêmico do comportamento dos veículos em seus deslocamentos sobre a malha viária para se fazer um controle efetivo que garanta melhorias como redução no número de acidentes de trânsito, redução no tempo de viagem dos motoristas e redução da emissão de poluentes devido à redução de consumo de combustível.

O avanço da tecnologia proporciona a possibilidade de implantação de métodos de controle nas vias, aumentando a capacidade das vias. Estudos atuais de automação veícular abordam ideias como veículos dirigitos automaticamente onde, por aquisição de dados por sensores e por comunicação com outros veículos e agentes no transito como semáforos, permitem que o veículo faça seu trajeto de forma mais segura, rápida e eficiênte. Existem algumas frentes com essa abordagem como controlador veicular centralizado (CVC) e rede veícular auto-roteada (VANET).

Situações incômodas no sistema de tráfego atual como a exposição ao perigo que os motoristas são obrigados a se submeter devido à violência noturna nas principais vias das cidades e vias congestionadas devido ao semáforo estar fechado sendo que a via transversal não possui veículos. Durante a noite, vias transversais vazias são motivações comuns para que motoristas cruzem o semáforo vermelho receosos devido ao perigo de esperar o semáforo abrir. Os semáforos em São Paulo não tem lógicas embarcadas para evitar esse tipo de situação. A tecnologia que soluciona essas situações já existe e foi implantada na cidade de Londres, Reino Unido, onde os semáforos se comunicam utilizando protocolo aberto, que possibilida a fabricação de semáforos com essa tecnologia por diversas empresas.

Diversos problemas reais tem soluções aplicadas a partir de um modelo \nocite{silveira2012universidade}  construido computacionalmente. Existem métodos para a modelagem de um problema por meio de tecnologias atuais, sendo possível simular a situação modelada e permitir o acompanhamento individual dos eventos discretizados no modelo, percebendo os pontos afetados pelo acontecimento do evento. Essa modelagem tem sido aplicada também no problema da degradação do tráfego urbano em grandes cidades metropolitanas, apresentando simuladores altamente parametrizáveis porém não tão visuais e intuitivos para utilização de usuários não especializados. 

\section{Objetivo}

O principal objetivo do trabalho é produzir um algoritmo para o funcionamento dos controladores de semáforos no tráfego urbano, onde esses controladores recuperam dados do sensoriamento colocado nas vias e trocam dados com os controladores mais próximos, e com base nos dados dos sensores e da comunicação, alteram a temporização de seus semáforos.

Para demonstrar o funcionamento desse algoritmo, será criada uma ferramenta visual de simulação de tráfego urbano, onde usuários com conhecimentos básicos de tráfego urbano podem criar mapas e parametrizações, e podem assistir em tempo real à simulação baseada nos dados fornecidos. Dois tipos de simulação poderão ser executados: Usando o algoritmo de otimização ou usando temporização fixa nos semáforos.

Serão geradas informações sobre a simulação para que o usuário perceba claramente a diferença entre os resultados dos dois tipos de simulação. Para melhor compreensão, serão exibidos gráficos comparando os principais indicadores dos trajetos, como tempo em movimento e tempo de espera.

\chapter{Teste qualquer}

Teste 1234Teste  1234Teste 1234Teste 1234Teste 1234Teste 1234Teste 1234Teste 1234Teste 1234Teste 1234Teste 1234Teste 1234 \nocite{schmidt2011arduino}

Teste 1234Teste 1234Teste 1234Teste 1234Teste 1234Teste 1234Teste 1234Teste 1234Teste 1234Teste 1234Teste 1234Teste 1234Teste 1234.

Teste 1234Teste 1234Teste 1234Teste 1234Teste 1234Teste 1234Teste 1234Teste 1234Teste 1234Teste 1234Teste 1234Teste 1234

Teste 1234Teste 1234Teste 1234Teste 1234Teste 1234Teste 1234Teste 1234Teste 1234Teste 1234Teste 1234Teste 1234Teste 1234Teste 1234

\bibliography{bibliografia}


\end{document}